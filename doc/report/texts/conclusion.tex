\section{Conclusion}

This project allowed to consolidate and put into practice the knowledge acquired during this semester in the specialization of Embedded Systems. In addition, this project required a lot of autonomous work, necessary to complement the concepts covered in the theoretical classes. This autonomous work allowed the learning of new skills that will be useful in the near future.

The development model used in this project was the Waterfall model, which defends the division of a project development into several stages, where each stage is only complete when the previous ones are also complete. In the first phase of Waterfall, the project plan, there were difficulties in defining the problem and in defining a concrete solution to that problem. In the analysis phase, the main problem focused on defining the type of communication to be used between the local system’s and a remote system. In addition, the planning of project activities also proved to be a challenge due to time constraints. All these difficulties delayed the project's progress, preventing the start of the design and implementation phases at the planned time.

During the implementation phase, several difficulties arose. The Buildroot settings for creating a custom Linux image took up a lot of implementation time. Setting up an image with OpenCV was the biggest difficulty, due to the lack of documentation on the associated Buildroot packages. The use of OpenCV libraries for the Raspberry Pi was a challenge, mainly in the compilation of OpenCV algorithms using Makefiles. Regarding the LoRa communication technology, it presented itself as a great challenge. Despite the use of third party software for this purpose, its suitability for the Raspberry Pi was the main issue. Initially, an API was used to communicate with the LoRa device, through SPI, which proved to be unsuitable for the purpose, making it difficult to identify the problem. As for the development of the mobile application, several problems arose, such as the installation of the IDE QtCreator for Android and also the implementation of the functionality to map the network lampposts on a map. The latter was not resolved, and another solution was developed for that purpose. Finally, the biggest problem became the lack of time to develop a project of this dimension, with only two elements.

Recalling the problem statement, the problem of the energy crisis associated with the existence of public lighting poles that are always on during the night, as well as the time of searching for parking spaces in a city, constitutes a problem for society. It is considered that the presented solution solves the mentioned problem, reducing the energy costs of public lighting, as well as reducing the search time for car parks in a city. However, for the full use of this solution, the pole must be installed in an area with nearby parking spaces. Furthermore, the remote system of this solution should be placed online, in order to be accessible anywhere in the world, as long as there is an internet connection.

\section{Future Work}
As with any project, there is room for improvement. This project is no exception and this section will present some improvements to be made to the project.

Regarding the local system, the main improvement would be to \textbf{improve the classifier} used in car detection, through the training of the Haar cascade classifier with images of cars in a different angle or the use of more recent object detection algorithms, like CNN or YOLO. The \textbf{encapsulation of the system} should also be improved, with better construction materials and adequate dimensions, facilitating the incorporation of this solution into the existing network of public lighting poles.

As for the gateway, the main improvement would be the addition of \textbf{encryption in the LoRa communication}. In the solution presented, anyone using a LoRa module operating in the 433 MHz frequency range will be able to access the messages exchanged between the remote system and the various local system’s.

The remote system can also be improved, starting with changing its \textbf{execution environment} to the cloud. The algorithm that allows the \textbf{dynamic control} of the lampposts should also be improved, through the use of a better way of identifying the lampposts that must be connected.

The remote clients website and mobile application are also subject to improvements, starting with the \textbf{implementation of a map} when consulting the lampposts network in the mobile application. In addition, the feasibility study of \textbf{including parking spaces information in an application} such as Google Maps, replacing the website solution, thus making it easier for users to consult this information.

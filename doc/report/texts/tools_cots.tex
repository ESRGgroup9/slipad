\clearpage
\section{Tools}

\begin{itemize}
	\item \textbf{Git:} free and open source distributed version control system;
	\item \textbf{GitHub:} provider of Internet hosting for software development and version control using Git;
	\item \textbf{Buildroot:} Tool to configure and generate the Raspberry Pi Kernel image;
	\item \textbf{C/C++:} Programming language used to develop local system and remote system core;
	\item \textbf{Qt Creator:} Cross-platform IDE used for the Mobile Application development;
	\item \textbf{HTML:} Programming language chosen for the WebSite development;
	\item \textbf{PHP:} Programming language used to fetch data from the database to the WebSite;
%	\item \textbf{Python:} Programming language chosen for the Haar cascade training stage;
	\item \textbf{MySQL:} Relational database management system used for the remote server database;
	\item \textbf{Makefile:} Used to compile the developed programs;
	\item \textbf{CMakeList:} Used to compile OpenCV algorithms;
	\item \textbf{Sublime Text:} Text editor used for coding;
	\item \textbf{Doxygen:} Used for generating documentation from annotated C/C++ sources;
\end{itemize}

\section{COTS}

\begin{itemize}
	\item \textbf{POSIX Threads API:} Used for thread creation and management;
	\item \textbf{OpenCV API:} Used for image capture and processing;
	\item \textbf{Qt API:} Used for the GUI;
%	\item \textbf{RaspiCam API:} C++ API for using Raspberry camera with/without OpenCv;
%	\item \textbf{Google Cloud SQL:} Google Cloud service that allows for immutable data storage and retrieval;
\end{itemize}

\clearpage
\section{Third-Party Libraries}

\begin{itemize}
	\item \textbf{Light Sensor (TSL258x) Device Driver:} Open-source device driver used for interfacing with the luminosity sensor \cite{code_tsl};
	\item \textbf{LoRa SX1278 Library:} An Arduino open-source library for sending and receiving data using LoRa radios \cite{sx1278_lib}. This is implemented to the Arduino board, but it will be adapted to the Raspberry Pi 4 Model B.
	\item \textbf{PHP dotenv:} Loads environment variables from \verb|.env| file, providing an easy way to load custom configuration variables that an application needs. \cite{phpdotenv}
\end{itemize}

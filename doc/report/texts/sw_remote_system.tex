%**********************************************************
The remote system is responsible for the communication between the local systems and the internet. In this project, the internet is used to maintain a cloud database, so the remote system does the data bridge between the network of street lampposts and a cloud. 

%As one can see in figure \ref{fig:rs_overview}, the remote system is composed by the main process and a daemon process \textit{dServerSend}. The main process may be composed of a server and varoius clients, being the messages from each client received and processed in this process. The messages are sent to the clients in the daemon process \textit{dServerSend}. The communication between the two processes is done using message queues, so when a command is received and successfully parsed, the main process sends the command output to the daemon using the message queue.

%\begin{figure}[H]
%	\centering
%	\includegraphics[width=.6\textwidth]{09sw_specification/RS/overview}
%	\caption{Inter-process Communication between Main Process and Daemon.}
%	\label{fig:rs_overview}
%\end{figure}

%**********************************************************
\subsection{Class Diagram}
In figure \ref{fig:rs_classDiag} is represented the class diagram of the remote system. The class \textit{CRemoteSystem} is the main class of this system, that initializes the objects of each class, listed below. As members, this class has a server, \textit{CTCPServer}, a list of connected clients, \textit{CClients}, and a database, \textit{CDataBase}. 

\begin{itemize}
	\item \textbf{CTCPServer:} responsible for creating the server and accepting the clients connections;
	\item \textbf{CClient:} client class, stores the clients information, receives their messages, executing the respective command and sends its output to the client;
	\item \textbf{CDataBase:} provides an interface between the server and the database.
\end{itemize}

\begin{figure}[H]
	\centering
	\includegraphics[width=.9\textwidth]{09sw_specification/RS/ClassDiagram}
	\caption{Remote System Class Diagram.}
	\label{fig:rs_classDiag}
\end{figure}


%****************************
\myparagraph{Class CTCPServer}

This class is responsible for creating the server, using \textit{createServer} method. Furthermore, using the function \textit{acceptConnection}, this class accepts the clients connections returning the socket file descriptor. This class contains the information about the server address, \textit{addr}, the listen socket descriptor, \textit{listenSd} and the maximum number of connected clients, \textit{maxClients}. In figure \ref{fig:CTCPServer}, one can see the class \textit{CTCPServer} diagram.

\begin{figure}[H]
	\centering
	\includegraphics[width=.6\textwidth]{09sw_specification/RS/CTCPServer/CTCPServer}
	\caption{Class Diagram: CTCPServer.}
	\label{fig:CTCPServer}
\end{figure}

%****************************
\myparagraph{Class CClient}

The class \textit{CClient}, represented in figure \ref{fig:CClient}, is responsible for storing information about a client connected to the remote system, like the \textit{cmdList}, a vector that stores the commands list that the client can execute and \textit{clientSock}. The \textit{clientSock} has information about the client like its \textit{state}, \textit{name}, identification number, \textit{index}, sock file descriptor, \textit{sockFd} and \textit{type}, that can be \textit{APPLICATION}, for a mobile application client, \textit{WEBSITE}, for a web site client or \textit{GATEWAY}, for a gateway device client. Furthermore, this class inherits from the class \textit{CCommunication}, that implements the mechanism of sending messages to the client, and implements the thread to receive and parse messages from the client, \textit{tRecv}. 

\begin{figure}[H]
	\centering
	\includegraphics[width=.9\textwidth]{09sw_specification/RS/CClient/CClient}
	\caption{Class Diagram: CClient.}
	\label{fig:CClient}
\end{figure}

%****************************
\myparagraph{Class CDataBase}

This class diagram is shown in figure \ref{fig:CDataBase}, and is responsible for implement the interface between the server and the database, having for that a pointer to a database of type \textit{MYSQL}. To interact with the database, the server must first prepare a query, using the function \textit{prepareQuery}, and can execute the prepared query in the database using the method \textit{execute}, that returns an error and assigns the command output to the string \textit{outMessage}.

\begin{figure}[H]
	\centering
	\includegraphics[width=.5\textwidth]{09sw_specification/RS/CDataBase}
	\caption{Class Diagram: CDataBase.}
	\label{fig:CDataBase}
\end{figure}

%**********************************************************
\subsection{Task Overview}
One can define and describe briefly how the remote system is implemented, making use of threads and processes. 

\begin{itemize}
	\item \textbf{CClient::tSend:} sends a message to the client;
	\item \textbf{CClient::tRecv:} receives a message from the client.
\end{itemize}

%**********************************************************
\subsection{Task Priority}
The priority assignment for the remote system is shown in figure \ref{fig:rsPrio}. 

\begin{figure}[H]
	\centering
	\includegraphics[width=.5\textwidth]{09sw_specification/RS/ThreadsPrio}
	\caption{Remote System Main Process Priority Assignment.}
	\label{fig:rsPrio}
\end{figure}

%**********************************************************
\subsection{Task Synchronization}
As seen before, to have coordinate access to shared ressouces and services and avoid race conditions, the kernel has resources that provide synchronization between the tasks. 

\myparagraph{Mutexes}

The mutexes used for the remote system are listed below.

\begin{itemize}
	\item \textbf{CCommunication::mutComms:} already detailed in the local system section; used in-directly by \textit{CClient}; used to protect the communications;
	\item \textbf{CCommunication::mutTxMsgs:} already detailed in the local system section; used in-directly by \textit{CClient}; used to protect the insertion and removal of messages into \textit{TxMsgs};
\end{itemize}

\myparagraph{Condition Variables}

The condition variables used for the remote system are listed below.

\begin{itemize}
	\item \textbf{CCommunication::condSend:} already detailed in the local system
	section; this condition variable is used indirectly by \textit{CClient}; used to notify CCommunication::tSend that a new message is ready to be sent.
\end{itemize}

%\subsection{Task Communication}
% Nothing to show here

%**********************************************************
%\clearpage
\subsection{Commands and Constants}
\myparagraph{Constants}

The constants used in the remote system are listed below.
\begin{itemize}
	\item \verb|MAX_CLIENT_NUM|: defines the maximum number of clients supported by the TCP server;
	\item \verb|SERVER_PORT|: defines the port in which the server will listen for client connections;
\end{itemize}

%*******************************
\myparagraph{Remote Server Commands}

The commands used for the communication between the remote server and the local systems, through the gateway, are listed below.
\begin{itemize}
	\item \verb|IDRQ|: ID request; Used to obtain a list of local systems connected to each gateway;
	\item \verb|<lsID> ON|: Turn on the lamp at full brightness of the local system with ID \verb|<lsID>|;
\end{itemize}

The commands used for the communication between the remote server and the remote client WebSite, are listed below.
\begin{itemize}
	\item \verb|PARK <num> <GPScoord>|: Send \verb|<num>| of available parking spots in the GPS coordinates, \verb|<GPScoord>|.
\end{itemize}

\clearpage
The commands used for the communication between the remote server and the remote client Mobile Application, are listed below.
\begin{itemize}
	\item \verb|OP <op_result>|: Presents to the mobile application the last operation result, \verb|<op_result>|.
	
	\item \verb|LS <lsID> <status> <GPScoord>|: Sends to the mobile application the information of a single local system, regarding the local system ID, current status and its GPS coordinates.
\end{itemize}

%*******************************
\myparagraph{Remote Client Commands}
The commands used for the communication between the remote client WebSite and the remote server, are listed below.
\begin{itemize}
	\item \verb|GET <streetName>|: request to obtain parking spots information in a street, \verb|<streetName>|.
\end{itemize}

The commands used for the communication between the remote server and the remote client Mobile Application, are listed below.
\begin{itemize}
	\item \verb|ADD <street> <postCode> <parish> <county> <district> <GPScoord>|: add a new lamppost to the network, giving its street name, post code, parish, county, district and GPS coordinates;
	\item \verb|REP <lsID> <street> <postCode>|: repair of a lamppost, with ID \verb|<lsID>|, at the a given street and post code. When receiving this command, the remote server changes the status of the lamppost to OFF.
\end{itemize}

%**********************************************************
\clearpage
\subsection{Start-up Process}
The start-up process is shown in the figure \ref{fig:RSstart}. When the program initiates, the start-up process instantiate a object of the class \textit{CRemoteSystem}. After that, it uses the method \textit{run} that will be seen in the next subsection.

\begin{figure}[H]
	\centering
	\includegraphics[width=0.3\textwidth]{09sw_specification/RS/start}
	\caption{Start-Up Process.}
	\label{fig:RSstart}
\end{figure}

%**********************************************************
\clearpage
\subsection{Flowcharts}
%********************************
\myparagraph{CRemoteSystem}

The figure \ref{fig:rsConst} represents the \textit{CRemoteSystem} class's constructor that is responsible for the creation of the \textit{clientList} vector, the database object, \textit{db} and the server \textit{server} in the port passed by argument.

\begin{figure}[H]
	\centering
	\includegraphics[width=.26\textwidth]{09sw_specification/RS/CRemoteSys/Const}
	\caption{Flowchart: CRemoteSystem Constructor.}
	\label{fig:rsConst}
\end{figure}

In figure \ref{fig:rsRun} is shown the \textit{CRemoteSystem} run method. Is is used to, after the creation of the server socket, accept the connections coming from the clients, using the \textit{CTCPServer} method to accept connections, \textit{acceptConnection}, that returns the socket descriptor of the new connection. If the socket descriptor \textit{sd} is valid, then it is created a client \textit{CClient} with this variable and the client is inserted in the client vector, \textit{clientList}.

\begin{figure}[H]
	\centering
	\includegraphics[width=.6\textwidth]{09sw_specification/RS/CRemoteSys/Run}
	\caption{Flowchart: CRemoteSystem run.}
	\label{fig:rsRun}
\end{figure}

%********************************
\myparagraph{CTCPServer}

In figure \ref{fig:ServerConst} is represented the constructor of the \textit{CTPCServer} class, that is responsible for creating the server socket in the port passed by argument for a maximum number of clients, \textit{maxClient}. This constructor creates a server socket with the IPv4 protocol connection, initialize the \textit{addr} with the port number passed by argument, binds the socket and listen to a maximum number of clients connections specified by \textit{maxClient} passed by argument.

\begin{figure}[H]
	\centering
	\includegraphics[width=.26\textwidth]{09sw_specification/RS/CTCPServer/Const}
	\caption{Flowchart: CTCPServer Constructor.}
	\label{fig:ServerConst}
\end{figure}

In figure \ref{fig:acceptConn} is represented the \textit{acceptConnection} method, that accepts a connection from a client in the socket referenced by the socket file descriptor \textit{listenSd}, returning the file descriptor of the new connection, \textit{sd}.

\begin{figure}[H]
	\centering
	\includegraphics[width=.25\textwidth]{09sw_specification/RS/CTCPServer/accept}
	\caption{Flowchart: CTCPServer accept.}
	\label{fig:acceptConn}
\end{figure}

%**********************************
\myparagraph{CClient}

In figure \ref{fig:ClientConst} is represented the constructor of the \textit{CClient} class. Every time a client connects to the remote system, it creates an instance of this class, passing by argument, the information about the client to initialize the \textit{clientSock} variable. The constructor also creates the command list that the type of client can execute, \textit{cmdList}, and the threads \textit{tRecv} and \textit{tSend}, using the method \textit{initThFun}.

\begin{figure}[H]
	\centering
	\includegraphics[width=.20\textwidth]{09sw_specification/RS/CClient/Const}
	\caption{Flowchart: CClient Constructor.}
	\label{fig:ClientConst}
\end{figure}

In the method \textit{initThFun}, shown in figure \ref{fig:clientInit}, it is created the thread \textit{tRecv} with priority \textit{recvPrio}, that receives messages from the clients, and the thread \textit{tSend} with priority \textit{sendPrio}, that is implemented in the class \textit{CCommunication} and sends messages to the clients.

\begin{figure}[H]
	\centering
	\includegraphics[width=0.26\textwidth]{09sw_specification/RS/CClient/init}
	\caption{Flowchart: CClient initThFun.}
	\label{fig:clientInit}
\end{figure}

The method \textit{runThFun}, represented in figure \ref{fig:clientRun}, is responsible for waiting for the termination of the threads \textit{tRecv} and \textit{tSend}.

\begin{figure}[H]
	\centering
	\includegraphics[width=0.26\textwidth]{09sw_specification/RS/CClient/run}
	\caption{Flowchart: CClient runThFun.}
	\label{fig:clientRun}
\end{figure}

The figure \ref{fig:RSRecv} shows the thread that receive messages from the clients, \textit{tRecv}. When a message is received, the function \textit{recv} returns the received message. If it is not empty, then one can parse and execute the command extracted from the received message, using the \textit{parseExecute} function. When the command is valid, this function returns a message to be sent back to the client. In order to send this message to the client, one needs to use the \textit{CCommunication} method to push this message to the sending buffer.

\begin{figure}[H]
	\centering
	\includegraphics[width=0.4\textwidth]{09sw_specification/RS/CClient/tRecv}
	\caption{Flowchart: CClient tRecv.}
	\label{fig:RSRecv}
\end{figure}

In figure \ref{fig:parseExecute} is represented the flowchart of the \textit{parseExecute} function, that is responsible for parsing and execute the message sent by the client. It is needed to verify if the client that sent the command has rights to execute that command, so the command sent is searched in the \textit{cmdList} assigned to the respective client. If the message corresponds to a valid command for that client, then one can execute the command. If the command doesn't exist in the command list of that client, then the \textit{msg} is assigned with an error message. The output of this function is the string message to be sent back to the client, \textit{msg}.

\begin{figure}[H]
	\centering
	\includegraphics[width=0.6\textwidth]{09sw_specification/RS/CClient/parseExecute}
	\caption{Flowchart: CClient parseExecute.}
	\label{fig:parseExecute}
\end{figure}

As for the previous subsystems, this class must implement the \textit{recvFunc} and \textit{sendFunc} methods, as they are pure virtual methods from CCommunication. 

In this class, one will use existent functions \textit{TCPReceive} and \textit{TCPSend} to implement TCP-IP communication, as shown in figures \ref{fig:RSrecvfunc} and \ref{fig:RSsendfunc}. This functions make use of the socket created when the TCP-IP connection was established.

\begin{figure}[H]
	\centering
	\includegraphics[width=.35\textwidth]{09sw_specification/RS/CClient/Recvfunc}
	\caption{Flowchart: CClient recvFunc method.}
	\label{fig:RSrecvfunc}
\end{figure}

\begin{figure}[H]
	\centering
	\includegraphics[width=.35\textwidth]{09sw_specification/RS/CClient/Sendfunc}
	\caption{Flowchart: CClient sendFunc method.}
	\label{fig:RSsendfunc}
\end{figure}

%**********************************************************
\clearpage
\subsection{Test Cases}
It is important to know how the remote server works, and how it will behave upon certain events. In table \ref{table:test_rs} are shown test cases to this system.

\begin{table}[H]
	\centering
	\begin{threeparttable}
	\resizebox{\columnwidth}{!}
	{
		\begin{tabular}{|m{4cm}|m{5cm}||m{5cm}|}
			\hline
			\textbf{Test Case} & \textbf{Expected Output} & \textbf{Real Output}
			\\\hline\hline
			\multicolumn{3}{c}{\textbf{Database Management}}\\\hline
			Open DB & DB opened and ready to use & As expected
			\\\hline
			Get data from DB & Receive requested data & As expected
			\\\hline
			Insert/Update/Remove data into DB & Insert/Update/Remove successful & As expected
			\\\hline
			
			\multicolumn{3}{c}{\textbf{TCP-IP Communications}}\\\hline
			Establish connection with RC & Connection established & As expected
			\\\hline
			Identify the RC & RC type and ID obtained & As expected
			\\\hline
			Process commands & Execute command if valid & As expected
			\\\hline
			
			\multicolumn{3}{c}{\textbf{Gateway Communications}}\\\hline
			Receive message from a gateway & Identify the referent LS & As expected
			\\\hline
			Send message to a specific gateway & Only the defined gateway receives the message & -
			\\\hline
			Send message to a group of gateways & All in the group receive the message & -
			\\\hline
			Send message to the gateway connected to a specific LS & Only the specified LS receives the message & -
			\\\hline
			Receive lamp ON command from a LS & Send same command to the neighbors LS & As expected
			\\\hline
			
			\multicolumn{3}{c}{\textbf{WebSite Communications}}\\\hline
			Get number of parking spots in a given location & DB returns number of parking spots in that location, if existent in the DB & As expected
			\\\hline
			
			\multicolumn{3}{c}{\textbf{Mobile App. Communications}}\\\hline
			Operator Login & Verify credentials & As expected
			\\\hline
			Add/remove/update lamppost info & Verify new info and insert into DB if valid & As expected
			\\\hline
		\end{tabular}
	}
	
		\begin{tablenotes}
			\small
			\item[DB]Database
			\item[LS]Local System
			\item[RS]Remote Server
			\item[RC]Remote Client
		\end{tablenotes}
	\end{threeparttable}

	\caption{Test Cases: Remote server.}
	\label{table:test_rs}
\end{table}

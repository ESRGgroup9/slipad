%**********************************************************
\section{Tools Setup}
Before doing the system configuration it is necessary to first setup all of the used tools, as will be next presented.

%**********************************************************
\subsection{Git}
In order to make collaboration easier, allowing change by multiple people to all be merged into one source, Git will be used. Git is the most commonly used version control system. Before using it, it is necessary to do the correct setup as shown bellow.
\begin{lstlisting}
$ sudo apt install git
$ git config --global user.name "John Doe"
$ git config --global user.email johndoe@example.com
$ git config --global core.editor subl
$ cd ~
$ git clone git@github.com:ESRGgroup9/slipad.git
\end{lstlisting}

With this steps, Git is installed in a local machine, username and user email is defined alongside with the default core editor. After this, one can clone the repository for this project, created in GitHub.

%**********************************************************
\subsection{Buildroot}
Buildroot is a simple, efficient and easy-to-use tool used to generate this project's embedded Linux system, through cross-compilation. The steps in order to install Buildroot in a local machine is shown bellow.

\begin{lstlisting}
$ cd ~
$ mkdir buildroot
$ cd buildroot
$ wget https://buildroot.org/downloads/buildroot-2021.02.5.tar.gz
$ tar xzf buildroot-2021.02.5.tar.gz
$ cd buildroot-2021.02.5
\end{lstlisting}

After the installation is done, one can do the base configurations, essential to the support the rest of the configurations.
\begin{lstlisting}
$ make raspberrypi4_defconfig
$ make menuconfig
$ make xconfig
$ make 
$ make clean
\end{lstlisting}

The first command is used to configure a kernel image for the Raspberry Pi 4, as it does the necessary configurations regarding hardware handling along with fetching some board specific packages. Then with the second and third commands, one can generate the graphic interface seen in figure \ref{fig:menuconfig}, presenting several sub-menus. 

\begin{figure}[H]
	\centering	
	\includegraphics[width=.74\textwidth]{11configs/menuconfig}
	\caption{Buildroot menuconfig.}
	\label{fig:menuconfig}
\end{figure}

%**********************************************************
\subsection{Qt}

%**********************************************************
\subsection{MySQL}
In order to have a fully managed database service to deploy cloud-native applications as MySQL, one has to firstly install it properly, as shown bellow.

\begin{lstlisting}
$ sudo apt-get install libmysqlclient-dev
\end{lstlisting}

After first run of \verb|mysql|, on can create insert Slipad's database and populate it, by typing the following commands.
\begin{lstlisting}
$ mysql -u root -p
mysql> source ~/slipad/database/slipad.sql
mysql> source ~/slipad/database/slipadPop.sql
\end{lstlisting}




%**********************************************************
\clearpage
\section{System Configuration}
In the Buildroot, using \verb|make menuconfig| one can use a graphic interface (as previously shown in figure \ref{fig:menuconfig}) to select packages and change configurations in order to create a custom linux image.\\

Under \verb|Target-Packages| select:
\begin{itemize}
	\item \verb|Development tools|, select:
	\begin{itemize}
		\item \verb|git|
	\end{itemize}

	\item \verb|Networking applications|, select:
	\begin{itemize}
		\item \verb|arp-scan|: scan the network of a certain interface for alive hosts;
		\item \verb|arptables-legacy|: set up and maintain the tables of ARP rules;
		\item \verb|dhcpcd|: open source DHCP client;
		\item \verb|dropbear|: small SSH server which will let us log in remotely;
	\end{itemize}

	\item \verb|Hardware Handling|
	\begin{itemize}
		\item \verb|spi-tools|: tools needed to use SPI. Used for LoRa module communication;
		\item \verb|i2c-tools|: tools needed to use I2C. Used for light sensor module, TSL2581;
	\end{itemize}
	
	\item \verb|Libraries -> Hardware Handling|, select:
	\begin{itemize}
		\item \verb|bcm2835|: C library for Raspberry Pi user programs;
	\end{itemize}
\end{itemize}

%**********************************************************
\clearpage
\section{Image Generation}
After all the necessary tools, packages and configurations are done, one needs to finally create the Linux custom image, that will run on the Raspberry Pi.

\begin{lstlisting}
$ cd ~/buildroot/buildroot-2021.02.5/
$ make
\end{lstlisting}

After that one can copy the image into the SD card, that will go into the Raspberry Pi. For that one can use the linux \verb|dd| command. \cite{dd}

\begin{lstlisting}
$ sudo dd if=./output/images/sdcard.img of=/dev/sdb
\end{lstlisting}

With this command, one must define the input file, \verb|if|, which is the linux image, and the output file, \verb|of|, which is the SD card, being in this example named \verb|sdb|.


%**********************************************************
%\clearpage
\section{System Initialization}
In the \verb|/boot/config.txt| file are some configuration parameters that are read when the system boots from the microSD card.

blablabla

%**********************************************************
\clearpage
\section{Device Drivers}
%\subsection{tsl2581}
%%.config - Linux/arm 5.10.1 Kernel Configuration
%%> Device Drivers > Industrial I/O support > Light sensors
%%
%%-> TAOS TSL2580, TSL2581 and TSL2583 light-to-digital converters
%%
%%cat /lib/modules/5.10.1-v7l/modules.dep | grep tsl
%In order to insert the \verb|tsl2581| device driver, available on the Linux kernel (\cite{code_tsl}) one needs to execute a series of steps.\\
%
%Firstly, one needs to add a new entry on the Raspberry Pi device tree, regarding the \verb|tsl2581|. For that, one needs to change the Device tree Source file (.dts) for the Raspberry Pi, named \verb|bcm2711-rpi-4-b.dts|. This will be later compiled into a device tree binary (.dtb), when creating an image using Buildroot, which will be later passed by the boot loader to the operating system kernel. \cite{dtb}
%
%\begin{lstlisting}
%$ cd ~/buildroot/buildroot-2021.02.5/output/build/linux-custom/arch/arm/boot/dts
%$ nano bcm2711-rpi-4-b.dts
%\end{lstlisting}
%
%At the end of the file, before \verb|__overrrides__|, the following code must be added, where \verb|reg| is the I2C address of the device. \cite{tsl2583_txt} 
%
%\begin{lstlisting}
%&i2c1 {
%	status = "okay";
%	
%	tsl2581@29 {
%		compatible = "amstaos,tsl2581";
%		reg = <0x29>;
%	};
%};
%\end{lstlisting}
%
%For the Raspberry Pi, in the \verb|/boot/config.txt| file, one must enable \verb|i2c1| with a \verb|dtoverlay|, by adding the line of code shown next.
%\begin{lstlisting}
%dtoverlay=i2c1
%\end{lstlisting}
%
%When booting the Raspberry Pi, one must insert the \verb|tsl2581| device driver, but before that, some modules must be added using \verb|modprobe|.
%
%This is an I2C based sensor, which uses the \ac{iio} interface. In order to provide the \ac{iio} API, needed in the device driver to be used, it is necessary to enable the \ac{iio} kernel subsystem, \verb|industrialio|. To enable the I2C bus, one must load the \verb|i2c-bcm2835| module. \cite{i2c_bcm2835}\\
%
%After that, one can insert the \verb|tsl2581| device driver, \verb|ldr.ko|.
%
%\begin{lstlisting}
%$ modprobe industrialio
%$ modprobe i2c-bcm2835
%$ insmod ldr.ko
%\end{lstlisting}
%
%After the device driver insertion, one can use it to communicate with the sensor in order to acquire the ambient luminosity. For that, a “single” on-demand read can be issued by user-space directly by reading \linebreak \verb|/sys/bus/devices/iio:device/in_<type><index>_raw|. In this case, the \verb|read_raw()| callback should handle basically all the steps necessary to get the required measurement. \cite{read_tsl}
%**********************************************************
\subsection{PIR}

%**********************************************************
\subsection{Lamp Failure Detector}

%**********************************************************
\clearpage
\section{LoRa communication}
LoRa communication was implemented by deriving a third-party software from Arduino. \cite{sx1278_lib} Using \verb|bcm2835| C library one was able to control all the needed GPIO and SPI functions. \cite{bcm2835} \cite{bcmspi}

By reading the LoRa module SX1278 documentation \cite{sx1278}, (page 80) one is able to find the figure \ref{fig:sx1278_spi}. The SPI interface gives access to the configuration register via a synchronous full-duplex protocol. One of three access modes to the registers is used - SINGLE access. In this mode:
\begin{itemize}
	\item \textbf{Write access:} an address byte followed by a data byte is sent;
	\item \textbf{Read access:} an address byte is sent and a read byte is received;
\end{itemize}

\begin{figure}[H]
	\centering	
	\includegraphics[width=1\textwidth]{12implementation/sx1278_spi}
	\caption{SPI Timing Diagram (single access).}
	\label{fig:sx1278_spi}
\end{figure}

A transfer is always started by the NSS pin going low. MISO is high impedance when NSS is high. MOSI is generated by the master on the falling edge of SCK and is sampled by the slave on the rising edge of SCK. MISO is generated by the slave on the falling edge of SCK. Both data to be transmitted and that has been received are stored in a configurable \ac{fifo} device. It is accessed via the SPI interface and provides several interrupts for transfer management. (Page 66, \cite{sx1278})
\\
In listing \ref{lst:lorasingletx} one can see LoRa single transfer function, which sends two bytes to the slave. 

\clearpage
\begin{lstlisting}[caption={LoRa single transfer.}, label={lst:lorasingletx}]
// set NSS pin low. Begin transfer
digitalWrite(_ss, LOW);

bcm2835_spi_transfer(address);
response = bcm2835_spi_transfer(value);

// set NSS pin high. Stop transfer
digitalWrite(_ss, HIGH);
\end{lstlisting}

By default, the device is configured at power up so that half of the available memory is dedicated to receive
(\verb|RegFifoRxBaseAddr| initialized at address 0x00) and the other half is dedicated for  (\verb|RegFifoTxBaseAddr| initialized at address 0x80). Therefore, when one wants to perform a transmit to the device, the address is always above \verb|0x80|. On the other hand, when one wants to perform a reading, the address is bellow \verb|0x80|. With that in mind one can use bit-masking to define the address for the operation, where, \verb|reg_addr| is the SX1278 register one wants to access:

\begin{itemize}
	\item \textbf{Read:} \verb|address = reg_addr & 0x7f| and \verb|value = 0x00|. The \verb|response| variable is the response from the slave;
	
	\item \textbf{Write:} \verb+address = reg_addr | 0x80+ and \verb|value| is the value to be written to the given address. \verb|response| is not used.
\end{itemize}

A LoRa message is defined by the class \verb|LoRaMsg| having the attributes shown in listing \ref{lst:loramsg}.

\begin{lstlisting}[caption={LoRa message.}, label={lst:loramsg}]
int recvAddr;     	// receiver address
int sendAddr;     	// sender address

int msgID;        	// message ID
size_t msgLength; 	// message length
string msg;       	// message
\end{lstlisting}

In listing \ref{lst:lorasend} is shown the main core of LoRa send function. This sends a series of attributes in each message, regarding destination address, sender address, message ID, message length and the message itself.

\begin{lstlisting}[caption={LoRa send function.}, label={lst:lorasend}]
beginPacket();

// add destination address  
write(destination);
// add sender address
write(localAddress);
// add message ID
write(msgCount);
// add message length
write(msg.length());
// add message
write(msg);

endPacket();
\end{lstlisting}

The function which implements LoRa receive, receives all of the attributes sent in the function above. Also it does some additional verifications to ensure communication integrity. The destination address in the message is compared to the local address of the device receiving the message, in order to avoid messages being mistakenly read. Besides that, one checks if the received message length matches the supposed length, by comparing the received message length to the message field regarding the message length.

In order to define LoRa status before a send/receive operation, a class is defined, as presented in listing \ref{lst:loraerr}.
 
\begin{lstlisting}[caption={LoRaError enum class.}, label={lst:loraerr}]
enum class LoRaError
{
	MSGOK = 0,  	// Message OK
	ENOMSGR,    	// No message received
	ENOTME,     	// Message received is not for this device
	EBADLMSG    	// Message received lengths does not match
};
\end{lstlisting}

%**********************************************************
\clearpage
\section{PWM control}
In order to control the lamp, a PWM signal is used. For that, one can use \verb|bcm2835| library to control a PWM channel, producing the desired PWM signal at the selected GPIO pin. \cite{bcmpwm}

Considering that the clock which drives the PWM channels, \verb|clock|, is 54 MHz, and the lamp will operate at 50 Hz, \verb|freq|.

\[ RANGE = \frac{clock}{clock_{div} * freq} \]

Using a clock divider of 16, $clock_{div}$, one gets \verb|RANGE = 67500|. The variable \verb|RANGE| defines the maximum range of the PWM output, as shown in listing \ref{lst:pwmconfig}. One used \verb|PWM_CHANNEL 0|.

\begin{lstlisting}[caption={PWM configuration.}, label={lst:pwmconfig}]
// Set the output pin to Alt Fun 5, to allow PWM channel 0 to be output there
bcm2835_gpio_fsel(PWM_PIN, BCM2835_GPIO_FSEL_ALT5);
// set clock divider
bcm2835_pwm_set_clock(BCM2835_PWM_CLOCK_DIVIDER_16);
//CTL reg
bcm2835_pwm_set_mode(PWM_CHANNEL, 1, 1);
//RNG1/2 reg
bcm2835_pwm_set_range(PWM_CHANNEL, RANGE);
\end{lstlisting}

One can set the PWM pulse ratio to emit to \verb|RANGE|, where the duty cycle, \verb|duty| a value from 0 to 1, controls the PWM output ratio as a fraction of the range as shown in listing \ref{lst:pwmset}.

\begin{lstlisting}[caption={PWM set duty cycle.}, label={lst:pwmset}]
bcm2835_pwm_set_data(PWM_CHANNEL, (duty*RANGE));
\end{lstlisting}

\clearpage
\section{Luminosity Sensor}
\label{section:lumSensor}
As specified before, in order to know when to turn on/ off the lamp, it is used a luminosity sensor, the TSL2581. This module uses the I2C communication protocol to interface with the Raspberry Pi, being implemented based on a third-party software \cite{tsl2581_code}. As for the LoRa communication, one used the bcm2835 C library to communicate with the sensor module by I2C. \cite{bcmpiic}

From the theoretical foundations, one knows that the TSL2581 module uses a data line, SDA, to send and receive data, and a clock signal line, SCL, to synchronize the communication with the master device.

The sensor datasheet \cite{TSL2581_DS} shows that for the command code byte is used the \textit{Command Register} of the TSL2581, that is composed by:

\begin{itemize}
	\item \textit{CMD} bit: set to '1' when one wants to select the \textit{COMMAND} register;
	\item \textit{TRANSACTION} two bits: selects type of transaction to follow in subsequent data transfers, read/ write (R/W) mode and the special function register. For '00', select the I2C writing mode. For '10', selects the I2C reading mode which supports the read block protocol. For 11, select the special function register. 
	\item \textit{ADDRESS} five bits: when the \textit{TRANSACTION} field selects the R/W mode, then this field selects the bus address of the slave; when the \textit{TRANSACTION} field is set to '11', this field specifies a special command function.
\end{itemize}	

\myparagraph{I2C Address}

This module has three 7-bit address, as shown in table \ref{table:tsl_address}, and one can select one of them. By default, the \textit{ADDR state} register is set to FLOAT, so this device address in the I2C bus is the 0x39.

\begin{table}[H]
	\centering
	\begin{tabular}{|m{3cm}|m{3cm}|}
		\hline
		\textbf{ADDR State} & \textbf{Address}
		\\\hline\hline
		VCC & 0x49
		\\\hline
		FLOAT & 0x39
		\\\hline
		GND & 0x29
		\\\hline
	\end{tabular}
	
	\caption{TSL2581 I2C addresses.}
	\label{table:tsl_address}
\end{table}

\myparagraph{TSL2581 Initialization}

After connecting the device in the I2C bus, one should configure its registers before reading the value from its two ADC channels. In the listing \ref{lst:tslConfig} is shown the TSL2581 device initialization. First, one needs to configure the I2C bus, using the \verb|DEV_ModuleInit| function, passing, by argument, the device address of the light sensor, \verb|ADDR_FLOAT| (0x39, default). Using the \verb|CONTROL| register, one can power on the device, that puts all registers to their default value, enabling the R/W transactions. As the day-night transition is slow, one does not need to read the luminosity value much frequently, so one can define the maximum integration time (688,5 ms). It is also important to define the gain of the sensor. This sensor supports gains of 1, 8, 16 or 111. For indoor tests, it was used a gain of 16, but in a real outdoor model, it must be used a gain of 1.

\begin{lstlisting}[caption={TSL2581 Initialization.}, label={lst:tslConfig}]
if(DEV_ModuleInit(ADDR_FLOAT)==1)
	return false;

IIC_Write(COMMAND_CMD | CONTROL,CONTROL_POWERON);	// power on

IIC_Write(COMMAND_CMD | TIMING, INTEGRATIONTIME_688MS);  	// 688,5 ms
IIC_Write(COMMAND_CMD | CONTROL, ADC_EN | CONTROL_POWERON); // Every ADC cycle generates interrupt
IIC_Write(COMMAND_CMD | INTERRUPT, INTR_INTER_MODE);		// TEST MODE
IIC_Write(COMMAND_CMD | ANALOG, GAIN_16X);					// GAIN = 16
\end{lstlisting}

\myparagraph{Read Luminosity Values}

When the device is configured, one can read the luminosity values of the module channels. The TSL2581 has two channels: 

\begin{itemize}
	\item Channel 0: Value of visible and infrared light;
	\item Channel 1: Value of infrared light.
\end{itemize}

The listing \ref{lst:tslRead} presents the code that reads the channel 0 values, as for the channel 1 values, it is the same procedure. The lower byte is read out first using the \verb|DATA0LOW| register and next the higher byte, using the \verb|DATA0HIGH| register. Then, one combine them into a 16-bit variable as seen in the line 3.

\begin{lstlisting}[caption={TSL2581 Channel 0 read.}, label={lst:tslRead}]
DataLow = IIC_Read(COMMAND_CMD | TRANSACTION | DATA0LOW); 	// read channel 0 low byte
DataHigh = IIC_Read(COMMAND_CMD | TRANSACTION | DATA0HIGH);	// read channel 0 high byte
Channel_0 = 256 * DataHigh + DataLow ;
\end{lstlisting}

This values should be now converted to the luminous intensity. First, one needs to scale the value using the formula presented in the listing \ref{lst:tslscale}.

\begin{lstlisting}[caption={TSL2581 Channel 0 scaling.}, label={lst:tslscale}]
// scale the channel values
channel0 = (Channel_0 * chScale0) >>  CH_SCALE;
\end{lstlisting}

After scaling, it is calculated the ratio between channel 0 and channel 1 using the formula shown in listing \ref{lst:tslscale}.

\begin{lstlisting}[caption={TSL2581 ratio calculation.}, label={lst:tslratio}]
ratio1 = (channel1 << (RATIO_SCALE + 1)) / channel0;
ratio = (ratio1 + 1) >> 1; // round the ratio value
\end{lstlisting}

After calculate the multiple of the relative channel using the formulas given in the datasheet, one can export the current luminous intensity through the code presented in listing \ref{lst:tslCalc}, where lux is the final luminosity value. 

\begin{lstlisting}[caption={TSL2581 luminosity value calculation.}, label={lst:tslCalc}]
temp = ((channel0 * b) - (channel1 * m));
temp += (1 << (LUX_SCALE - 1));			// round lsb (2^(LUX_SCALE-1))
//  temp = temp + 32768;
lux_temp = temp >> LUX_SCALE;			// strip off fractional portion
\end{lstlisting}


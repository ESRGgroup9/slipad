The Buildroot generates the boot configuration, in the \verb|/boot/config.txt| file that contains system configuration parameters that are read when the system boots from the microSD card, and that would traditionally be edited and stored using a BIOS. \cite{configtxt} The listing \ref{lst:configtxt} shows the boot configuration file. For this project, one added the parameter in line 24, in order to enable the I2C device, the parameters in lines 26 and 27, to activate the camera.

\begin{lstlisting}[caption={/boot/config.txt file.}, label={lst:configtxt}]
# We always use the same names, the real used variant is selected by
# BR2_PACKAGE_RPI_FIRMWARE_{DEFAULT,X,CD} choice
start_file=start.elf
fixup_file=fixup.dat

kernel=zImage

# To use an external initramfs file
#initramfs rootfs.cpio.gz

# Disable overscan assuming the display supports displaying the full resolution
# If the text shown on the screen disappears off the edge, comment this out
disable_overscan=1

# How much memory in MB to assign to the GPU on Pi models having
# 256, 512 or 1024 MB total memory
gpu_mem_256=100
gpu_mem_512=100
gpu_mem_1024=100

# fixes rpi (3B, 3B+, 3A+, 4B and Zero W) ttyAMA0 serial console
dtoverlay=miniuart-bt

dtparam=i2c_arm=on 		# enable i2c

start_x=1             	# enable camera
gpu_mem=128           
\end{lstlisting}

%**********************************************************
\subsection{Init Scripts}
\verb|/etc/init.d| is a directory containing initialization and termination scripts for changing init states. File names in this directory are of the form \verb|[SK]nn<init.d filename>|, where \verb|S| means start this job, \verb|K| means kill this job, and \verb|nn| is the relative sequence number for killing or starting the job. \cite{initScript} For this reason, one must create a script, with name started by \verb|S|, inside \verb|/etc/init.d| in the Raspberry Pi, using for that, \verb|vi|.

In order to guarantee that each program is running after the system boots up, its necessary to create the respective init scripts, following the procedure detailed in listing \ref{lst:initScript}.

\begin{lstlisting}[caption={Procedure for creating an init script, named Sscript\_name}, label={lst:initScript}]
$ cd /etc/init.d
$ vi S<scriptName>.sh
$ chmod +x S<scriptName>.sh
\end{lstlisting}

The local system has the init script detailed bellow, in listing \ref{lst:lsScript}. This script must insert all necessary device drivers in order to correctly execute the local system's daemon, dSensors, and the main process. 

\begin{lstlisting}[caption={Init script for local system.}, label={lst:lsScript}]
#!/bin/sh

# Inserting i2c-tools
modprobe i2c-bcm2835
modprobe i2c-dev

# Inserting PIR DDriver
insmod /etc/pir.ko

# Inserting lampf DDriver
insmod /etc/lampf.ko

# Run local system's daemon sensor and main process
/etc/dSensors.elf &
/etc/localSys.elf
\end{lstlisting}

The gateway system has the init script detailed in listing \ref{lst:gatScript}. The gateway, \verb|gateway.elf| is connected via TCP-IP to the remote system. Therefore one can set the gateway to connect to \verb|10.42.0.1|, which is the network address of remote system in the same network as the gateway, in port \verb|5000|.

\begin{lstlisting}[caption={Init script for Gateway system.}, label={lst:gatScript}]
#!/bin/sh

etc/gateway.elf 10.42.0.1 5000
\end{lstlisting}

The remote system also has an init script, as shown in listing \ref{lst:rsScript}. This is responsible for running the website PHP server \cite{phpserver} and running the remote system in port \verb|5000|.

\begin{lstlisting}[caption={Init script for remote system.}, label={lst:rsScript}]
#!/bin/sh

# run website PHP server
php -S localhost:8000 -t ~/slipad/remoteSystem/website/ &

# run remote system
~/slipad/remoteSystem/bin/remoteSystem.elf 5000

\end{lstlisting}
%\subsection{tsl2581}
%%.config - Linux/arm 5.10.1 Kernel Configuration
%%> Device Drivers > Industrial I/O support > Light sensors
%%
%%-> TAOS TSL2580, TSL2581 and TSL2583 light-to-digital converters
%%
%%cat /lib/modules/5.10.1-v7l/modules.dep | grep tsl
%In order to insert the \verb|tsl2581| device driver, available on the Linux kernel (\cite{code_tsl}) one needs to execute a series of steps.\\
%
%Firstly, one needs to add a new entry on the Raspberry Pi device tree, regarding the \verb|tsl2581|. For that, one needs to change the Device tree Source file (.dts) for the Raspberry Pi, named \verb|bcm2711-rpi-4-b.dts|. This will be later compiled into a device tree binary (.dtb), when creating an image using Buildroot, which will be later passed by the boot loader to the operating system kernel. \cite{dtb}
%
%\begin{lstlisting}
%$ cd ~/buildroot/buildroot-2021.02.5/output/build/linux-custom/arch/arm/boot/dts
%$ nano bcm2711-rpi-4-b.dts
%\end{lstlisting}
%
%At the end of the file, before \verb|__overrrides__|, the following code must be added, where \verb|reg| is the I2C address of the device. \cite{tsl2583_txt} 
%
%\begin{lstlisting}
%&i2c1 {
%	status = "okay";
%	
%	tsl2581@29 {
%		compatible = "amstaos,tsl2581";
%		reg = <0x29>;
%	};
%};
%\end{lstlisting}
%
%For the Raspberry Pi, in the \verb|/boot/config.txt| file, one must enable \verb|i2c1| with a \verb|dtoverlay|, by adding the line of code shown next.
%\begin{lstlisting}
%dtoverlay=i2c1
%\end{lstlisting}
%
%When booting the Raspberry Pi, one must insert the \verb|tsl2581| device driver, but before that, some modules must be added using \verb|modprobe|.
%
%This is an I2C based sensor, which uses the \ac{iio} interface. In order to provide the \ac{iio} API, needed in the device driver to be used, it is necessary to enable the \ac{iio} kernel subsystem, \verb|industrialio|. To enable the I2C bus, one must load the \verb|i2c-bcm2835| module. \cite{i2c_bcm2835}\\
%
%After that, one can insert the \verb|tsl2581| device driver, \verb|ldr.ko|.
%
%\begin{lstlisting}
%$ modprobe industrialio
%$ modprobe i2c-bcm2835
%$ insmod ldr.ko
%\end{lstlisting}
%
%After the device driver insertion, one can use it to communicate with the sensor in order to acquire the ambient luminosity. For that, a “single” on-demand read can be issued by user-space directly by reading \linebreak \verb|/sys/bus/devices/iio:device/in_<type><index>_raw|. In this case, the \verb|read_raw()| callback should handle basically all the steps necessary to get the required measurement. \cite{read_tsl}
%**********************************************************
\subsection{PIR}

%**********************************************************
\subsection{Lamp Failure Detector}
\section{Problem Statement}
Nowadays, the energy crisis is a constant theme because of the inflated energy prices \cite{energy_crisis}. Furthermore, huge energy consumption is a burden to the environment, as not all means of energy production are non-polluting. According to "Our World in Data"\cite{owidenergy}, in 2019, 63,3 \% of eletrical energy production comes from fossil fuels. It is known that generally, street lamps are continuously switched on at night, most of the time unnecessarily glowing with its full intensity, in the absence of any activities in the street, leading to a great waste of energy. Furthermore, it is in cities where the consequences of using cars are most noticeable. An example of this is the search for a parking space. According to the RAC Foundation \cite{cars_parked}, in England, an average car is parked 95 \% of the time, which explains how hard it can get sometimes when trying to find a parking spot. This struggle leads to an increase in carbon dioxide production as well as fuel and energy consumption.

With that in mind, this project aims the implementation of applications for a Smart City, regarding Smart Lighting and Smart Parking, in order to decrease the energy consumption in public streets, while improving the lives of citizens around the world. The solution will embrace a centralized system, composed by smart street lights capable of turning on only when they detect movement in the surroundings, at night time, and also, capable of detecting available parking spaces in the street post vicinity.

\clearpage
\section{Problem Statement Analysis}
This solution provides a network of street lamp posts, each implementing Smart Street Lighting and Smart Parking Detection, using Raspberry Pi 4B \cite{rasp_pi} has a controller. A gateway is needed to gather all the information from the street lamp posts, and store that in a remote system, needed to provide a way for a responsible entity manage the network.

When there is no activity detected in the area, the lamp post is at a predefined minimum light level, whereas when a car or pedestrian is noticed in the area, the light automatically activates at full brightness. To allow to dynamically turn on the lights of the following poles, each the street lamp post communicates with the neighbor lamp posts, indirectly, through the gateway. To detect movement in the vicinity of the pole, a motion detector is used. Since the lamppost will only light up during the night time, the motion detector will also only work during that period. To ensure this, a luminosity sensor is used, determining the ambient light conditions. In order to facilitate the maintenance of the pole, a system that determines the operating conditions of the lamp is also implemented. When this system verifies that the lamp is not in good working conditions, in other words, that it is broken or burnt, this information is transmitted to the entity responsible for the network of lamp posts, through a mobile app. This is also used by the person in charge, to manage all information on the pole network, such as the location and working conditions of each pole.

In order to detect empty parking spots, this system should only be used in an area where there are parking spaces nearby. For this, the lamp post has a camera, turned on all day, and, after Raspberry Pi processes the acquired information, it will be available on a website, so that a user, a car driver, can know where there are empty parking spaces.



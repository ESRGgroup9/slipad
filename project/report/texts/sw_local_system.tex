%**********************************************************
\subsection{Task Overview}
One can define and describe briefly how the local system is implemented, making use of threads and processes.

\begin{itemize}
	\item \textbf{tCamera:} acquire a camera frame; processes image and search for parking spots; verify the parking spot availability;
	\item \textbf{tLampControl:} initializes PWM interface; controls the PWM applied to the lamp;
	\item \textbf{tLoraSend:} sends a message to the gateway, using the LoRa module;
	\item \textbf{tLoraRecv:} receives a message from the gateway, using the LoRa module;
	\item \textbf{dReadSensors:} ???
\end{itemize}

%**********************************************************
\subsection{Task Priority}

%**********************************************************
\subsection{Task Synchronization}
Real-time tasks share resources and services, and as such, should be prepared to await for the availability of these resources and services, like logical resources (buffers and data), physical resources, services like directory services, etc. In order to have coordinate access to shared resources and avoid race conditions, the kernel has resources that provide synchronization tools. 

\myparagraph{Condition Variables}

A condition variable is a task synchronization tool that can be used to block (wait) one or more threads, suspending its execution. The blocked threads are awakened when the condition variable is notified. The condition variables used are listed below.

\begin{itemize}
	\item \textbf{condCameraAcquire:} used to notify \textit{tCamera} that a camera sample period has elapsed;
	
	\item \textbf{condNewPWM:} used to notify \textit{tLampControl} that a new PWM value for the lamp was defined;
	
	\item \textbf{condSend:} used to notify \textit{tLoraSend} that a new message is ready to be sent;
	
	
\end{itemize}

\myparagraph{Mutexes}
A mutex is a locking mechanism that provides mutual exclusion, supporting ownership and other protocols. A mutex is initially created in the unlocked state in which it can be acquired by a task. After being acquired, the mutex moves to the locked state. When the task releases the mutex, it returns to the unlocked state. The mutexes used are listed bellow.

\begin{itemize}
	\item \textbf{mutCamera:} mutex associated with the condition variable \textit{condCameraAcquire} to acquire a camera frame;
		
	\item \textbf{mutSend:} protects the message to be sent in \textit{tLoraSend}, which can be defined in multiple places;
	
	\item \textbf{mutComms:} protects LoRa communication, since it is half-duplex, so one can send or receive at a time; Used in \textit{tLoraSend} and \textit{tLoraRecv};
	
	\item \textbf{mutChangePWM:} protects the modification of \textit{pwm\_val} variable, when defining a new PWM value for the lamp;
	
	
\end{itemize}

\myparagraph{Semaphores}

\subsection{Task Communication}
\myparagraph{Message Queues}
\myparagraph{Signals}

%**********************************************************
\subsection{Flowcharts}
\clearpage
\myparagraph{tCamera}

The task tCamera is responsible for acquire a frame from the Raspberry Pi Camera and to process it. It must analyze the returned frame, in order to detect empty parking spots. This thread uses the mutex \textit{mutCamera} to protect the condition variable \textit{condCameraAcquire}, that synchronizes the camera frame acquisition with the timer that defines the camera frame acquisition period, \textit{timSampleCam}.

Firstly, this thread initializes the camera device, sets the timer \textit{timSampleCam}, locks the mutex \textit{mutCamera} and goes to sleep mode, waiting for the conditon variable \textit{condCameraAcquire} to be signaled. This happens when a \textit{timSampleCam} period has elapsed and the thread wakes up. Now, one can read a camera frame in order to process it, unlocking the mutex \textit{mutCamera}. A timer, \textit{timCamMax}, is setted to report the error if the image is taking too much time to being processed.

In the image processing part of the thread, if there aren't parking spots coordinates stored, then it is necessary to search for parking spots. After that, one can verify the existance of cars in the image using a function to detect objects

\myparagraph{tLampControl}
% uses: mutChangePWM; condNewPWM; pwm_val
% pwm_val is used in: PIR_sensor, CommandCb, LDR
% must define: MIN_BRIGHT_PWM; LAMP_ON_TIMEOUT
This task is responsible for initializing and controlling the PWM peripheral used to control the lamp brightness. A mutex \textit{mutChangePWM} is used to protect the process of defining a new PWM value. 

After initializing the PWM, this task goes to sleep, waiting for the condition variable \textit{condNewPWM} to notify this task. This happens when a new PWM value is defined into the variable \textit{pwm\_val}, in \textit{dReadSensors} or in a received command. Then, the lamp PWM is changed to the new value. 

The lamp may have various levels of luminosity: for the lamp to be OFF, is applied PWM~=~0; for the lamp to be at a predefined minimum bright level, is applied PWM~=~\textit{MIN\_BRIGHT\_PWM}; for the lamp to be at maximum bright, this is, when the lamp must be ON, is applied PWM~=~100. So, since the lamp must stay ON a minimum amount of time out of a motion is detected or out of a request from the remote system to be ON, one needs to check if the new PWM is the maximum value. If so, that means that the lamp should continue with that PWM for a predefined time, defined by \textit{LAMP\_ON\_TIMEOUT}, in seconds.

\myparagraph{tLoraSend}

\myparagraph{tLoraRecv}

%**********************************************************
\subsection{Start-up Process}

%**********************************************************
\subsection{Shutdown Process}


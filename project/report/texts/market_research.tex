\section{Market Research}
\subsection{Market Definition}
Public lighting is essential to the society quality of life, since it allows citizens to enjoy public spaces at night, providing greater security. “In 1417, the Mayor of London ordered all houses to hang lanterns outdoors after dark during the winter months. This marked the first organized public lighting.” \cite{street_lighting_history}. From oil lamps to \ac{led} lamps, public lighting has become a more efficient, cheaper and less polluting way of lighting the streets. 

Currently, most of the lamps used in public lighting are \ac{hps}. This is a gas-discharge lamp that uses sodium to produce light, at a distinctively yellow-orange, monochromatic glow. These are more efficient than the older incandescent lights, have a cheaper price and have a higher lumen efficiency than older street light types. However, these have a higher maintenance cost and operation cost than the \ac{led} lamps. Also, \ac{hps} lamps doesn’t have the advantage of being a directional light, like \ac{led} does, meaning that \ac{hps} light gets emitted in various directions, contributing to light pollution. \cite{led_vs_hps}

The market is driven by several factors, among which are regulatory policies, \ac{iot} convergence, and \ac{led} price, in addition to the culture and morphology of each area. \ac{led} technology can generate savings of more than 60 percent of energy costs \cite{light_pollution}, allowing payback of the initial investment. But, on the other hand, it comes with hidden costs: people tend to overuse it and over-illuminate areas, wasting energy unnecessarily by casting large amounts of light in all directions, emitting bluelight wavelengths that bounce around in the atmosphere, badly affecting animals, including humans.

\subsection{Market Dimension and Growth}

Cities are looking at smart infrastructure to reduce costs, improve sustainability, and provide better services to residents. Nowadays, Telensa is the market share leader in smart street lighting with more than ten years of experience. PLANet is an  intelligent street lighting system, consisting of wireless nodes connecting individual lights, a dedicated network owned by the city and a central management application. This system reduces energy and maintenance costs associated with street lighting and also improves quality of maintenance through automatic fault reporting. Doncaster, the largest metropolitan borough in England, houses over 45,000 smart Telensa streetlights, covering 220 square miles, achieving energy savings of approximately 1,5 million euros annually, with potential to increase this in the future \cite{telensa}.

FLASHNET is a company focused on developing intelligent systems for smarter cities and better infrastructures and have created a solution that provides the right amount of light where and when needed to lighten the streets, the inteliLIGHT \cite{inteli_light}. Using the existing infrastructure, this solution saves money and transforms the existing distribution level network into an intelligent infrastructure of the future. Furthermore, the system is integrated with major \ac{iot} platforms and provides \ac{api} connectivity with City Management applications, ensuring compatibility with existing smart lighting and smart city initiatives.

Note that smart street light is an emerging technology that, despite being established in the market, is still relatively uncommon in cities due to the initial investment it entails. However, it is clear that in the long run it is compensatory and now, as the consequences of light pollution tiptoe from the shadows and into the spotlight, cities, regulatory agencies, and conservation groups are agitating for solutions.
\section{Market Research}
\subsection{Market Definition}
Public lighting is essential to the society quality of life, since it allows citizens to enjoy public spaces at night, providing greater security. “In 1417, the Mayor of London ordered all houses to hang lanterns outdoors after dark during the winter months. This marked the first organized public lighting.” [ref]. From oil lamps to LED lamps, public lighting has become a more efficient, cheaper and less polluting way of lighting the streets. 

Currently, most of the lamps used in public lighting are High Pressure Sodium (HPS). This is a gas-discharge lamp that uses sodium to produce light, at a distinctively yellow-orange, monochromatic glow. These are more efficient than the older incandescent lights, have a cheaper price and have a higher lumen efficiency than older street light types. However, these aren’t directional lights (have 360 degrees light direction, which isn't ideal for light pollution), have a higher maintenance cost and operation cost than the LED lamps. 

The market is driven by several factors, among which are regulatory policies, IoT convergence, and LED price, in addition to the culture and morphology of each area. LED technology can generate savings of 50–80 percent of energy costs, allowing payback of the initial investment.


- Historia do poste
	- Tecnologia de um poste
- Tecnologias de lampadas
- Consumos
- Custos associados


\subsection{Market Dimension and Growth}
%When Los Angeles recently replaced more than 150,000 streetlights with LEDs, the city saved roughly 8 million dollars annually, or more than 60 percent on energy costs.
%Now, as the consequences of light pollution tiptoe from the shadows and into the spotlight, cities, regulatory agencies, and conservation groups are agitating for solutions.

%**********************************************************
\subsection{Task Overview}
One can define and describe briefly how the gateway is implemented, making use of threads and processes.

\begin{itemize}
	\item \textbf{tLoraRecv: } receives all messages from local systems, using LoRa communication;
	\item \textbf{tTCPRecv: } receives all messages from the remote server, using TCP-IP communication.	
	\item \textbf{tLoraSend: } sends all received messages from the remote server, in textit{tTCPRecv}, to the local systems, using LoRa communication;
	\item \textbf{tTCPSend: } sends all received messages from local systems, in \textit{tLoraRecv}, to the remote server, using TCP-IP communication;
\end{itemize}

One can define the relationship between the tasks as follows, in figure \ref{fig:gwOverview}. To communicate the local system received messages and the remote server sender service its used a vector of messages, named \textit{msgs\_to\_rs}. In the communication between the remote server received messages and the local system sender service its also used a vector of messages, named \textit{msgs\_to\_ls}.

\begin{figure}[H]
	\centering
	\includegraphics[width=1\textwidth]{09sw_specification/gwOverview}
	\caption{Gateway Overview.}
	\label{fig:gwOverview}
\end{figure}

%**********************************************************
\subsection{Task Priority}

%**********************************************************
\subsection{Task Synchronization}

\myparagraph{Condition Variables}
The condition variables used in this system are listed below.

\begin{itemize}
	\item \textbf{condLoraSend: }
	\item \textbf{condTCPSend: }
\end{itemize}

\myparagraph{Mutexes}
The mutexes used in this system are listed bellow.

\begin{itemize}
	\item \textbf{mutLoraComm: }
	\item \textbf{mutLoraSend: }
	\item \textbf{mutTCPComm: }
	\item \textbf{mutTCPSend: }
		
\end{itemize}

%\myparagraph{Semaphores}

\subsection{Task Communication}
In order to communicate all messages between the LoRa related tasks and the TCP related tasks are used two vectors of strings, \textit{msgs\_to\_rs} and \textit{msgs\_to\_ls}. This belongs to the container library \textit{std::vector} from C++ libraries, which implements a dynamic contiguous array. The storage of the vector is handled automatically being expanded and contracted as needed. This provides functions like \textit{push\_back()}, \textit{insert()} or \textit{pop\_back()} which allows to insert or remove new elements as needed.
%\myparagraph{Message Queues}
%\myparagraph{Signals}

%**********************************************************
\subsection{Flowcharts}
Each local system has a predefined ID, which may be presented in a physical label for an operator to use this information. When a local system is being installed, it will use it's predefined ID in all communications with the remote server until the operator registers the local system that's being installed, in the remote server. By doing this, the remote server assigns a new ID to the local system, which is then sent to the local system, for this to be used in further communications.

\myparagraph{tLoraRecv}
This task, presented in figure \ref{fig:gwtLoraRecv}, is responsible for receiving all the packets sent by the local systems, through LoRa communication. This task makes use of a mutex \textit{mutLoraComm}, to protect LoRa send and receive functions, which must not occur at the same time. Besides that, it uses a mutex \textit{mutTCPSend} to protect the vector of messages \textit{msgs\_to\_rs} which can be changed in \textit{tTCPRecv}.

When a message is received it is pushed into the vector of messages \textit{msgs\_to\_rs}, to be sent to the remote system. This way, we can receive continually messages from local systems, while there is another task sending this messages. This operation is protected by \textit{mutTCPSend}. After the message is added to the vector of messages, this task signals the task \textit{tTPCSend} through the condition variable \textit{condTCPSend}, in order for this to send the newly added message to the remote system.

\begin{figure}[H]
	\centering
	\includegraphics[width=.5\textwidth]{09sw_specification/gwtLoraRecv}
	\caption{Flowchart: Gateway tLoraRecv.}
	\label{fig:gwtLoraRecv}
\end{figure}

\myparagraph{tLoraSend}
This task, presented in figure \ref{fig:gwtLoraSend}, is responsible for sending all the packets sent by the remote system to the local systems. In addiction to the mutex \textit{mutLoraComm}, this task makes use of another mutex \textit{mutLoraSend}, which comes along with the condition variable \textit{condLoraSend}, in order to protect the insertion and removal of messages from the vector \textit{msgs\_to\_ls}.

When there are no messages to send to the local systems, i.e, the messages vector \textit{msgs\_to\_ls} is empty, then the task enters a sleep state, waiting for \textit{condLoraSend} to be notified. When this happens, the mutex that protects LoRa communication is locked and a message is popped from the queue of messages to being sent to the local systems. If the vector \textit{msgs\_to\_ls} is not empty after sending a message, this task continues to do so until there are no more to send, before entering into sleep state again.

\begin{figure}[H]
	\centering
	\includegraphics[width=.8\textwidth]{09sw_specification/gwtLoraSend}
	\caption{Flowchart: Gateway tLoraSend.}
	\label{fig:gwtLoraSend}
\end{figure}

\myparagraph{tTCPSend}

\myparagraph{tTCPRecv}
%**********************************************************
\subsection{Start-up Process}

%**********************************************************
\subsection{Shutdown Process}

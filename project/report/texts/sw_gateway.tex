%**********************************************************
\subsection{Task Overview}
One can define and describe briefly how the gateway is implemented, making use of threads and processes.

\begin{itemize}
	\item \textbf{tLoraRecv:}
	\item \textbf{tLoraSend:}
	\item \textbf{tTCPSend:} sends a message to the remote server, via TCP/IP communication protocol;
	\item \textbf{tTCPRecv:} receives a message from the remote server, via TCP/IP communication protocol.
\end{itemize}

%**********************************************************
\subsection{Task Priority}

%**********************************************************
\subsection{Task Synchronization}

\myparagraph{Condition Variables}
The condition variables used in this system are listed below.

\begin{itemize}
	\item \textbf{condLoraSend:}
	
	\item \textbf{condTCPSend:} used to notify \textit{tTCPSend} that a new message is ready to be sent;
		
\end{itemize}

\myparagraph{Mutexes}
The mutexes used in this system are listed bellow.

\begin{itemize}
	\item \textbf{mutLoraComm:}
	\item \textbf{mutLoraSend:}	
	
	\item \textbf{mutTCPComm:} mutex used to protect the TCP/IP communications (send and receive);
	\item \textbf{mutTCPSend:} mutex associated with the condition variable \textit{condTCPSend} to send a message via TCP/IP protocol communication;
\end{itemize}

%\myparagraph{Semaphores}

%\subsection{Task Communication}
%\myparagraph{Message Queues}
%\myparagraph{Signals}

%**********************************************************
\subsection{Flowcharts}
\myparagraph{tLoraRecv}

\myparagraph{tLoraSend}

\myparagraph{tTCPSend}

\myparagraph{tTCPRecv}
%**********************************************************
\subsection{Start-up Process}

%**********************************************************
\subsection{Shutdown Process}

To define the network architecture of the solution to be created, some aspects must be remembered. One is that there are various communication technologies that may be used, as presented previously in Market Research. Other important aspect to keep in mind is that this solution implements both Smart Street Lighting and Smart Parking, through the use of street lampposts. So, these must have parking spots nearby, in order to allow full use of the Smart Parking feature. This lack of flexibility demands a creation of a network with nodes that may be far apart. Besides that, the data stream in the network will be very low, that way, one can identify LoRa as a proper communication technology to use in this network.

\begin{figure}[ht]
	\centering
	\includegraphics[width=1\textwidth]{/03system_overview/network_arch}
	\caption{Network architecture.}
	\label{fig:network_arch}
\end{figure}

In figure \ref{fig:network_arch} one can see the network architecture diagram. This is a star topology, as the wireless communication takes advantage of the Long Range characteristics of LoRa, allowing a single-hop link between each lamp post, the local system, and the gateway. This one is connected to the internet in order to store new information in a remote system.



That being said, if each lamppost is spaced by 4 meters, each base station can easily connect with 10 local systems. To communicate with a remote server, the router will be connected to the internet through an Ethernet cable, or similar, that will most certainly already exist near the lampposts, in the telecommunications infrastructure.


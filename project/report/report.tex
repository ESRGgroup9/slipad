\documentclass[12pt, letterpaper]{article}
\usepackage[utf8]{inputenc}
\usepackage{graphicx}
\usepackage{indentfirst}
\usepackage{url}
\usepackage{titling}

\graphicspath{{./images/}}

\newcommand{\subtitle}[1]{%
  \posttitle{%
    \par\end{center}
    \begin{center}\LARGE#1\end{center}
    \vskip0.5em}%
}

\title{\textbf{Smart Street Lighting}}

\subtitle{{\large Master in Industrial Eletronics and Computers Engeneering} \\ {\large Embedded Systems}}

\author{Authors:\\Diogo Fernandes PG47150\\José Tomás Abreu PG47386\\ \\ Supervisors:\\Prof. Dr. Tiago Gomes\\Prof. Ricardo Roriz\\Prof. Sérgio Pereira}

\date{\today}

\begin{document}

{\begin{figure}[t]
	\centering
	\includegraphics[width=0.35\textwidth]{EEUMLOGO}
\end{figure}}

\maketitle

%\tableofcontents
\clearpage
\section{Problem Statement}
Nowadays, the energy crisis is a constant theme because of the inflated energy prices \cite{energy_crisis}. Furthermore, huge energy consumption is a burden to the environment, as not all means of energy production are non-polluting \cite{energy_supply}. It is known that, in cities, street lamps are continuously switched on at night, most of the time unnecessarily since there are many places not regularly used, which leads to a great waste of energy, also contributing to the increase in light pollution \cite{light_pollution}.
	
With that in mind, the main objective of this project is the creation of a monitoring device, capable of controlling an intelligent street lamps network. These are capable of turning on only when they detect movement in the surroundings, adjusting their luminosity according to the needs of the surrounding environment and monitor nearby traffic. The device to be developed must be able to connect to all the sensors of each street lamp, must have knowledge of each operating conditions and location, and also must be able to control the street lamps individually, if necessary.



\bibliographystyle{IEEEtran}
\bibliography{References}

\end{document}
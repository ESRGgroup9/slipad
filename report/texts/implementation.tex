%**********************************************************
\section{Tools Setup}

\subsection{Git}
aaaaaa
\begin{lstlisting}
$ sudo apt install git
$ git config --global user.name "John Doe"
$ git config --global user.email johndoe@example.com
$ git config --global core.editor subl
$ git clone git@github.com:ESRGgroup9/slipad.git
\end{lstlisting}

\subsection{Buildroot}
aaaaaa
\begin{lstlisting}
$ cd ~
$ mkdir buildroot
$ cd buildroot
$ wget https://buildroot.org/downloads/buildroot-2021.02.5.tar.gz
$ tar xzf buildroot-2021.02.5.tar.gz
$ cd buildroot-2021.02.5
\end{lstlisting}

aaaaaa 
\begin{lstlisting}
$ make raspberrypi4_defconfig
$ make menuconfig
$ make xconfig
$ make 
$ make clean
\end{lstlisting}

\subsection{Qt}

\subsection{MySQL}

\section{System Configuration}
In the Buildroot, using \verb|make menuconfig| one can use a graphic interface to select packages and change configurations in order to create a custom linux image for the system to be developed. Initially, this graphical interface presents sub-menus regarding different options to configure, as shown in figure \ref{fig:menuconfig}.

\begin{figure}[H]
	\centering	
%	\includegraphics[width=.5\textwidth]{11configs/}
	\caption{Buildroot menuconfig.}
	\label{fig:menuconfig}
\end{figure}

Under \verb|Target-Packages| select:
\begin{itemize}
	\item \verb|Hardware Handling -> spi-tools|: LoRa module communicates with the Raspberry Pi through SPI, therefore it is necessary to enable the protocol and the tools required to use it.
	\item \verb|Libraries -> Hardware Handling -> bcm2835|
	
\end{itemize}

%**********************************************************
\clearpage
\section{System Initialization}

%**********************************************************
\clearpage
\section{Device Driver}
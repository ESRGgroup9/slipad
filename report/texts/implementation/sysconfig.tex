In the Buildroot, using \verb|make menuconfig| one can use a graphic interface (as previously shown in figure \ref{fig:menuconfig}) to select packages and change configurations in order to create a custom linux image.\\

Under \verb|Target-Packages| select:
\begin{itemize}
	\item \verb|Development tools|, select:
	\begin{itemize}
		\item \verb|git|
	\end{itemize}

	\item \verb|Networking applications|, select:
	\begin{itemize}
		\item \verb|arp-scan|
		\item \verb|arptables-legacy|
		\item \verb|dhcpcd|
		\item \verb|dropbear|
	\end{itemize}

	\item \verb|Hardware Handling|
	\begin{itemize}
		\item \verb|spi-tools|: LoRa module communicates with the Raspberry Pi through SPI, therefore it is necessary to  enable the protocol and the tools required to use it.
	\end{itemize}
	
	\item \verb|Libraries -> Hardware Handling|, select:
	\begin{itemize}
		\item \verb|bcm2835|
	\end{itemize}	
\end{itemize}
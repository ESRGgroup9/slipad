%**********************************************************
\section{Tools Setup}
Before doing the system configuration it is necessary to first setup all of the used tools, as will be next presented.

%**********************************************************
\subsection{Git}
In order to make collaboration easier, allowing change by multiple people to all be merged into one source, Git will be used. Git is the most commonly used version control system. Before using it, it is necessary to do the correct setup as shown bellow.
\begin{lstlisting}
$ sudo apt install git
$ git config --global user.name "John Doe"
$ git config --global user.email johndoe@example.com
$ git config --global core.editor subl
$ cd ~
$ git clone git@github.com:ESRGgroup9/slipad.git
\end{lstlisting}

With this steps, Git is installed in a local machine, username and user email is defined alongside with the default core editor. After this, one can clone the repository for this project, created in GitHub.

%**********************************************************
\subsection{Buildroot}
Buildroot is a simple, efficient and easy-to-use tool used to generate this project's embedded Linux system, through cross-compilation. The steps in order to install Buildroot in a local machine is shown bellow.

\begin{lstlisting}
$ cd ~
$ mkdir buildroot
$ cd buildroot
$ wget https://buildroot.org/downloads/buildroot-2021.02.5.tar.gz
$ tar xzf buildroot-2021.02.5.tar.gz
$ cd buildroot-2021.02.5
\end{lstlisting}

After the installation is done, one can do the base configurations, essential to the support the rest of the configurations.
\begin{lstlisting}
$ make raspberrypi4_defconfig
$ make menuconfig
$ make xconfig
$ make 
$ make clean
\end{lstlisting}

The first command is used to configure a kernel image for the Raspberry Pi 4, as it does the necessary configurations regarding hardware handling along with fetching some board specific packages. Then with the second and third commands, one can generate the graphic interface seen in figure \ref{fig:menuconfig}, presenting several sub-menus. 

\begin{figure}[H]
	\centering	
	\includegraphics[width=.74\textwidth]{11configs/menuconfig}
	\caption{Buildroot menuconfig.}
	\label{fig:menuconfig}
\end{figure}

%**********************************************************
\subsection{Qt}

%**********************************************************
\subsection{MySQL}
In order to have a fully managed database service to deploy cloud-native applications as MySQL, one has to firstly install it properly, as shown bellow.

\begin{lstlisting}
$ sudo apt-get install libmysqlclient-dev
\end{lstlisting}

After first run of \verb|mysql|, on can create insert Slipad's database and populate it, by typing the following commands.
\begin{lstlisting}
$ mysql -u root -p
mysql> source ~/slipad/database/slipad.sql
mysql> source ~/slipad/database/slipadPop.sql
\end{lstlisting}




%**********************************************************
\clearpage
\section{System Configuration}
In the Buildroot, using \verb|make menuconfig| one can use a graphic interface (as previously shown in figure \ref{fig:menuconfig}) to select packages and change configurations in order to create a custom linux image.\\

Under \verb|Target-Packages| select:
\begin{itemize}
	\item \verb|Development tools|, select:
	\begin{itemize}
		\item \verb|git|
	\end{itemize}

	\item \verb|Networking applications|, select:
	\begin{itemize}
		\item \verb|arp-scan|: scan the network of a certain interface for alive hosts;
		\item \verb|arptables-legacy|: set up and maintain the tables of ARP rules;
		\item \verb|dhcpcd|: open source DHCP client;
		\item \verb|dropbear|: small SSH server which will let us log in remotely;
	\end{itemize}

	\item \verb|Hardware Handling|
	\begin{itemize}
		\item \verb|spi-tools|: tools needed to use SPI. Used for LoRa module communication;
		\item \verb|i2c-tools|: tools needed to use I2C. Used for light sensor module, TSL2581;
	\end{itemize}
	
	\item \verb|Libraries -> Hardware Handling|, select:
	\begin{itemize}
		\item \verb|bcm2835|: C library for Raspberry Pi user programs;
	\end{itemize}
\end{itemize}

%**********************************************************
\clearpage
\section{Image Generation}
After all the necessary tools, packages and configurations are done, one needs to finally create the Linux custom image, that will run on the Raspberry Pi.

\begin{lstlisting}
$ cd ~/buildroot/buildroot-2021.02.5/
$ make
\end{lstlisting}

After that one can copy the image into the SD card, that will go into the Raspberry Pi. For that one can use the linux \verb|dd| command. \cite{dd}

\begin{lstlisting}
$ sudo dd if=./output/images/sdcard.img of=/dev/sdb
\end{lstlisting}

With this command, one must define the input file, \verb|if|, which is the linux image, and the output file, \verb|of|, which is the SD card, being in this example named \verb|sdb|.


%**********************************************************
%\clearpage
\section{System Initialization}
In the \verb|/boot/config.txt| file are some configuration parameters that are read when the system boots from the microSD card.

blablabla

%**********************************************************
\clearpage
\section{Device Drivers}
%\subsection{tsl2581}
%%.config - Linux/arm 5.10.1 Kernel Configuration
%%> Device Drivers > Industrial I/O support > Light sensors
%%
%%-> TAOS TSL2580, TSL2581 and TSL2583 light-to-digital converters
%%
%%cat /lib/modules/5.10.1-v7l/modules.dep | grep tsl
%In order to insert the \verb|tsl2581| device driver, available on the Linux kernel (\cite{code_tsl}) one needs to execute a series of steps.\\
%
%Firstly, one needs to add a new entry on the Raspberry Pi device tree, regarding the \verb|tsl2581|. For that, one needs to change the Device tree Source file (.dts) for the Raspberry Pi, named \verb|bcm2711-rpi-4-b.dts|. This will be later compiled into a device tree binary (.dtb), when creating an image using Buildroot, which will be later passed by the boot loader to the operating system kernel. \cite{dtb}
%
%\begin{lstlisting}
%$ cd ~/buildroot/buildroot-2021.02.5/output/build/linux-custom/arch/arm/boot/dts
%$ nano bcm2711-rpi-4-b.dts
%\end{lstlisting}
%
%At the end of the file, before \verb|__overrrides__|, the following code must be added, where \verb|reg| is the I2C address of the device. \cite{tsl2583_txt} 
%
%\begin{lstlisting}
%&i2c1 {
%	status = "okay";
%	
%	tsl2581@29 {
%		compatible = "amstaos,tsl2581";
%		reg = <0x29>;
%	};
%};
%\end{lstlisting}
%
%For the Raspberry Pi, in the \verb|/boot/config.txt| file, one must enable \verb|i2c1| with a \verb|dtoverlay|, by adding the line of code shown next.
%\begin{lstlisting}
%dtoverlay=i2c1
%\end{lstlisting}
%
%When booting the Raspberry Pi, one must insert the \verb|tsl2581| device driver, but before that, some modules must be added using \verb|modprobe|.
%
%This is an I2C based sensor, which uses the \ac{iio} interface. In order to provide the \ac{iio} API, needed in the device driver to be used, it is necessary to enable the \ac{iio} kernel subsystem, \verb|industrialio|. To enable the I2C bus, one must load the \verb|i2c-bcm2835| module. \cite{i2c_bcm2835}\\
%
%After that, one can insert the \verb|tsl2581| device driver, \verb|ldr.ko|.
%
%\begin{lstlisting}
%$ modprobe industrialio
%$ modprobe i2c-bcm2835
%$ insmod ldr.ko
%\end{lstlisting}
%
%After the device driver insertion, one can use it to communicate with the sensor in order to acquire the ambient luminosity. For that, a “single” on-demand read can be issued by user-space directly by reading \linebreak \verb|/sys/bus/devices/iio:device/in_<type><index>_raw|. In this case, the \verb|read_raw()| callback should handle basically all the steps necessary to get the required measurement. \cite{read_tsl}
%**********************************************************
\subsection{PIR}

%**********************************************************
\subsection{Lamp Failure Detector}

%**********************************************************
\clearpage
\section{LoRa communication}
LoRa communication was implemented by deriving a third-party software from Arduino. \cite{sx1278_lib} Using \verb|bcm2835| C library one was able to control all the needed GPIO and SPI functions. \cite{bcm2835}

By reading the LoRa module SX1278 documentation \cite{sx1278}, one is able to find the figure \ref{fig:sx1278_spi}. The SPI interface gives access to the configuration register via a synchronous full-duplex protocol. One of three access modes to the registers is used - SINGLE access. In this mode:
\begin{itemize}
	\item \textbf{Write access:} an address byte followed by a data byte is sent;
	\item \textbf{Read access:} an address byte is sent and a read byte is received;
\end{itemize}

\begin{figure}[H]
	\centering	
	\includegraphics[width=1\textwidth]{12implementation/sx1278_spi}
	\caption{SPI Timing Diagram (single access).}
	\label{fig:sx1278_spi}
\end{figure}

A transfer is always started by the NSS pin going low. MISO is high impedance when NSS is high. MOSI is generated by the master on the falling edge of SCK and is sampled by the slave on the rising edge of SCK. MISO is generated by the slave on the falling edge of SCK.

A LoRa message is defined by the class \verb|LoRaMsg| having the attributes shown in listing \ref{lst:loramsg}.

\begin{lstlisting}[caption={LoRa message}, label={lst:loramsg}]
int recvAddr;     	// receiver address
int sendAddr;     	// sender address

int msgID;        	// message ID
size_t msgLength; 	// message length
string msg;       	// message
\end{lstlisting}

In listing \ref{lst:lorasend} is shown the main core of LoRa send function. This sends a series of attributes in each message, regarding destination address, sender address, message ID, message length and the message itself.

\begin{lstlisting}[caption={LoRa send function}, label={lst:lorasend}]
beginPacket();

// add destination address  
write(destination);
// add sender address
write(localAddress);
// add message ID
write(msgCount);
// add message length
write(msg.length());
// add message
write(msg);

endPacket();
\end{lstlisting}

The function which implements LoRa receive, receives all of the attributes sent in the function above. Also it does some additional verifications to ensure communication integrity. The destination address in the message is compared to the local address of the device receiving the message, in order to avoid messages being mistakenly read. Besides that, one checks if the received message length matches the supposed length, by comparing the received message length to the message field regarding the message length.

In order to define LoRa status before a send/receive operation, a class is defined, as presented in listing \ref{lst:loraerr}.
 
\begin{lstlisting}[caption={LoRaError}, label={lst:loraerr}]
enum class LoRaError
{
	MSGOK = 0,  	// Message OK
	ENOMSGR,    	// No message received
	ENOTME,     	// Message received is not for this device
	EBADLMSG    	// Message received lengths does not match
};
\end{lstlisting}